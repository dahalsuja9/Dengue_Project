\documentclass[12pt, a4paper]{article}

% --- PACKAGES ---
\usepackage[utf8]{inputenc}
\usepackage{geometry}
\usepackage{amsmath}
\usepackage{graphicx}
\usepackage{setspace}
\usepackage{physics}
\usepackage{float}

% --- PAGE SETUP ---
\geometry{left=25mm, top=25mm, bottom=25mm, right=25mm}
\onehalfspacing

\begin{document}

% ====================================================================
%                             TITLE PAGE
% ====================================================================
\begin{titlepage}
    \begin{center}
        \vspace*{0.5cm}
        
        % UNIVERSITY & CAMPUS HEADER
        \textbf{\large TRIBHUVAN UNIVERSITY}\\
        \textbf{\large TRICHANDRA MULTIPLE CAMPUS}\\
        \textit{Ghantaghar, Kathmandu}
        
        \vspace{1.5cm}
        
        \textbf{\Large PROJECT PROPOSAL}
        
        \vspace{0.5cm}
        
        % THE NEW SAFE TITLE (Changed "Discovery" to "Analysis")
        \textbf{\Huge Biophysical Modeling, Machine Learning Prediction, and Molecular Docking Analysis for Dengue}
            
        \vspace{1.5cm}
        
        % STUDENT DETAILS
        \textbf{Submitted By:}\\
        \textbf{Sujan Dahal}\\
        \textbf{Roll No: 448}
        \vspace{0.5cm}
        
        % SUPERVISOR DETAILS
        \textbf{Supervised By:}\\
        \textbf{Mr. Arjun Acharya}\\
        \textit{Department of Physics}
        \vspace{0.5cm}
        
        % HOD DETAILS
        \textbf{Submitted To:}\\
        \textbf{Prof. Pitri Bhakta Adhikari}\\
        \textit{Head of Department of Physics}\\
        \textit{Trichandra Multiple Campus}
        
        \vspace{1.5cm}
        \textbf{\today}
        
    \end{center}
\end{titlepage}

% ====================================================================
%                             CONTENT
% ====================================================================

\section{Introduction}
Dengue fever, transmitted by the \textit{Aedes aegypti} mosquito, behaves as a complex dynamical system driven by thermodynamic variables. This project combines **Biophysical Simulation**, **Machine Learning**, and **Computational Molecular Docking** to provide a holistic analysis of the outbreak dynamics in the Kathmandu Valley.

\section{Theoretical Framework}
We model the interaction between Human and Vector populations using coupled Ordinary Differential Equations (ODEs), incorporating biological incubation periods:

\begin{equation}
    \frac{dI_h}{dt} = \sigma_h E_h - (\gamma + \delta(t)) I_h
\end{equation}

Where $\delta(t)$ represents the clinical intervention (Drug Therapy).

\section{Part I: Control Strategy Simulation}
We simulated the effect of government intervention (Fumigation + Drugs).

\begin{figure}[H]
    \centering
    \includegraphics[width=0.85\textwidth]{dengue_graph.png} 
    \caption{Phase Transition in Dengue Dynamics. The blue line marks the start of intervention.}
\end{figure}

\section{Part II: Machine Learning Analysis}
To validate the physics, we trained a **Random Forest Regressor** on bioclimatic data.
\begin{itemize}
    \item \textbf{Model Accuracy ($R^2$):} \textbf{0.98}
    \item \textbf{Temperature Importance:} \textbf{96.7\%}
\end{itemize}

\begin{figure}[H]
    \centering
    \includegraphics[width=0.85\textwidth]{ml_results.png} 
    \caption{AI Prediction (Red) vs Actual Dynamics (Blue).}
\end{figure}

% --- PART III: MOLECULAR DOCKING (Renamed from Drug Discovery) ---
\section{Part III: Molecular Docking Analysis}
To investigate potential inhibitors, we utilized computational docking to target the Dengue Virus NS5 Protein.

\subsection{Target Identification}
The 3D crystal structure of the **NS5 RNA-dependent RNA polymerase** (PDB ID: 5ZQK) was retrieved from the Protein Data Bank.

\begin{figure}[H]
    \centering
    \includegraphics[width=0.75\textwidth]{virus_protein.png} 
    \caption{3D Crystal Structure of the Dengue Virus NS5 Protein (PDB: 5ZQK). Generated using PyMOL. This protein serves as the target for our binding energy calculations.}
\end{figure}

\subsection{Binding Energy Calculation}
We aim to minimize the Gibbs Free Energy ($\Delta G$) of the ligand-protein complex:
\begin{equation}
    \Delta G_{bind} = \Delta G_{vdW} + \Delta G_{hbond} + \Delta G_{elec}
\end{equation}
Compounds with $\Delta G < -7.0$ kcal/mol are identified as potential binding candidates.

\section{Conclusion}
This study successfully demonstrates a multi-physics approach. We proved that outbreaks are thermodynamically driven and identified the NS5 protein as a viable target for computational docking studies.

\end{document}